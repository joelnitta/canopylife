% set style for units and abbreviations
% for New Phytologist

%%% SI units
\usepackage{siunitx}

% over-ride SI units following journal style
\DeclareSIUnit{\hour}{h}
\DeclareSIUnit{\second}{s}

% make custom units
\DeclareSIUnit{\ma}{MA}
\DeclareSIUnit{\year}{yr}
\DeclareSIUnit{\month}{mo}
\DeclareSIUnit{\week}{wk}
\DeclareSIUnit{\day}{d}
\DeclareSIUnit{\basepair}{bp}
\DeclareSIUnit[space-before-unit=true]{\species}{spp.}

% commas between digits for large numbers, starting with 10 thousands
\sisetup{group-separator = {,}}
\sisetup{group-minimum-digits = 5}

% detect bold font
\sisetup{detect-weight,mode=text}

% use en dash for range, don't repeat units for each number
%\sisetup{range-phrase = --}
%\sisetup{range-units = single}

%%% Abbreviations
% custom commands for abbreviations that can vary from journal to journal.
% use package xspace to control spacing
% (note that \xspace can't be nested)
% see https://tex.stackexchange.com/questions/31091/space-after-latex-commands

\usepackage{xspace}
% Figures
\newcommand{\fig}{Fig.\xspace}
\newcommand{\figs}{Figs\xspace}
% Tables
\newcommand{\tab}{Table\xspace}
\newcommand{\tabs}{Tables\xspace}
% Diameter
\newcommand{\diameter}{diam.\xspace}
% Latin
\newcommand{\sensulato}{s. l.\xspace}
\newcommand{\sensustricto}{s. s.\xspace}
\newcommand{\ie}{i.e.\xspace}
\newcommand{\eg}{e.g.\xspace}
\newcommand{\sensu}{\textit{sensu}\xspace}
\newcommand{\circa}{\textit{c.}\xspace}

% Statistics
% Sample size
\newcommand{\n}{\textit{n}\xspace}
% P value
\newcommand{\pval}{\textit{P}\xspace}
% R2 value
\newcommand{\rval}{\textit{R}\xspace}
% T test
\newcommand{\tval}{\textit{t}\xspace}
% Z score
\newcommand{\zval}{\textit{Z}\xspace}
% degrees of freedom
\newcommand{\df}{df\xspace}
% F value
\newcommand{\fval}{\textit{F}\xspace}

% Standard deviation
\newcommand{\stdev}{SD\xspace}
% Standard error
\newcommand{\stderr}{SE\xspace}

% Obscure statistics
% Blomberg's K
\newcommand{\K}{\textit{K}\xspace}
% Fritz Purvis D
\newcommand{\D}{\textit{D}\xspace}

% Computer commands / functions
\newcommand{\function}[1]{`{#1}'\xspace}
% R packages
\newcommand{\package}[1]{`{#1}'\xspace}